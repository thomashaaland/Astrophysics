\documentclass[a4paper,10pt]{article}
\usepackage{amsmath}
\usepackage{graphicx}
\usepackage{verbatim}
\usepackage[latin1]{inputenc}
\usepackage{fancyvrb}
\DefineVerbatimEnvironment{code}{Verbatim}{fontsize=\small}
\DefineVerbatimEnvironment{example}{Verbatim}{fontsize=\small}

% Shortcuts for including equations
\newcommand{\beq}{\begin{equation}}
\newcommand{\eeq}{\end{equation}}
\def\i{\hat{i}}
\def\j{\hat{j}}
\def\k{\hat{k}}

% Document formatting
\setlength{\parindent}{0mm}
\setlength{\parskip}{1.5mm}

% Hyper refs
\usepackage[colorlinks]{hyperref}

\usepackage{listings}
\lstset{language=python}
\lstset{basicstyle=\ttfamily\tiny}
\lstset{frame=single}
\lstset{stringstyle=\ttfamily}
\lstset{keywordstyle=\color{red}\bfseries}
\lstset{commentstyle=\itshape\color{blue}}
\lstset{showspaces=false}
\lstset{showstringspaces=false}
\lstset{showtabs=true}
\lstset{breaklines}
\lstdefinestyle{prt}{frame=none,basicstyle=\ttfamily\small}

\newcounter{subproject}
\renewcommand{\thesubproject}{\number{subproject}}
\newenvironment{subproj}{
\begin{description}
\item[\refstepcounter{subproject}(\thesubproject)]
}{\end{description}}

%------------------------------------------------------------
\begin{document} 
{\huge\bf Ast1100 Oblig 6}
\newline Thomas Haaland


Den kosmiske avstandsstigen

\newpage

{\bf Oppgave 1}
\newline
En stjerne blir observert $1''$ skifte over himmelen over et halvt �r. Skal finne parallax vinkelen og distansen. 

Parallax vinkelen er definert som $\theta$ p� figur 1. Her er $\theta$ halvparten av den totale vinkelen som blir m�lt, alts� $\theta=\frac{1}{2}''$. $d$ er distansen til objektet, og $1AU$ er distansen fra jorda til sola, som blir Basislinjen. 

\begin{figure}
 \centering
 \includegraphics[scale=0.3]{figure1.pdf}
 \label{fig: parallax}
 \caption{parallax angle}
\end{figure}
 
Distansen $d$ blir dermed

\beq
d=\frac{1AU}{\theta}=\frac{1}{\frac{1}{2}''}=2pc
\eeq

som f�lge av definisjonen av parsec.


{\bf Oppgave 2}
\newline
Igjen bruker definisjonen av parsec

\beq
\theta''=\frac{1AU}{d}pc
\eeq

M� konvertere lys�r til $pc$: $\frac{4.22ly}{3.262\frac{ly}{pc}}=1.3pc$. Satt inn blir $\theta$

\beq
\frac{1AU}{1.3pc}=0.7''
\eeq


{\bf Oppgave 3}

Velger gul stjerne med $T=6000K$. Ser da at liten $m$ gir liten $M$. Leser da av HR-diagrammet at dersom $m=6$, burde $M=1$. 

Satt inn i formelen for � finne rekkevidden fra magnitude

\beq
M_V-m_V=-5\log\frac{d}{10pc}
\eeq

\beq
10^{M_V-m_V}=\left(\frac{10pc}{d}\right)^5
\eeq

\beq
d=\frac{10pc}{\left(10^{M_V-m_V}\right)^{\frac{1}{5}}}=\frac{10pc}{\left(10^{1-6}\right)^{\frac{1}{5}}}=100pc
\eeq


{\bf Oppgave 4}

Ser at supernova type 1a har $M_V\simeq-19.3^+_-0.3$. Dette er kjent p� grunn av Chandrasekhar grensen.

Bruker dette for � finne distansen som i forrige oppgave:

\beq
d=\frac{10pc}{\left(10^{M_V-m_V}\right)^{\frac{1}{5}}}=\frac{10pc}{\left(10^{-19.3-20}\right)^{\frac{1}{5}}}=7.2E+8pc
\eeq


{\bf Oppgave 5}

Her m� Hubbles lov brukes. Da m� $v$ f�rst finnes. Bruker dopplers formel:

\beq
\frac{\lambda-\lambda_0}{\lambda_0}=\frac{v}{c}
\eeq

\beq
v=\frac{\lambda-\lambda_0}{\lambda_0}c=\frac{29.7cm-21.2cm}{21.2cm}c=0.4c
\eeq

Hastigheten funnet her er p� grunn av romtidsdilasjon, og effekt fra spesiell relativitet kan dermed sees bort i fra.

Bruker Hubbles lov:

\beq
r=\frac{v}{H_0}=\frac{0.4c}{71\frac{km}{s}/Mpc}=1690Mpc
\eeq


{\bf Oppgave 6}
Formel for relasjon mellom observert magnitude $m$ og absolutt magnitude $M$ korrigert for optisk dybde er

\beq
m_V-M_V=5\log_{10}\left(\frac{r}{10pc}\right)+1.086\tau
\eeq

L�ser for $r$ og f�r

\beq
r=\frac{10pc}{\sqrt[5]{10^{M_V-m_V+1.086\tau}}}=\frac{10pc}{\sqrt[5]{10^{1-6+1.086\times0.2}}}=36.8pc
\eeq


{\bf Oppgave 7}

Skal finne ny distanse til supernovaen over der vi ogs� tar hensyn til optisk dybde. L�ser p� nytt med formelen for relasjon mellom observert magnitude og absolutt magnitude korrigert for optisk dybde: 

\beq
r_{ny}=\frac{10pc}{\sqrt[5]{10^{M_V-m_V+1.086\tau}}}=\frac{10pc}{\sqrt[5]{10^{-19.3-20+1.086\times1}}}=4.4E+8
\eeq

feilen blir da $r-r_{ny}=7.2E+8-4.4E+8=2.8E+8$
\end{document}
